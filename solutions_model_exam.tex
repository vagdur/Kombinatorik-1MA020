\documentclass[nobib]{tufte-handout}

\title{Lösningsförslag för tentamen i kombinatorik, 15 Mars 2023 $\cdot$ 1MA020}

\author[Vilhelm Agdur]{Vilhelm Agdur\thanks{\href{mailto:vilhelm.agdur@math.uu.se}{\nolinkurl{vilhelm.agdur@math.uu.se}}}}

\date{15 mars 2023}


%\geometry{showframe} % display margins for debugging page layout

\usepackage{graphicx} % allow embedded images
  \setkeys{Gin}{width=\linewidth,totalheight=\textheight,keepaspectratio}
  \graphicspath{{graphics/}} % set of paths to search for images
\usepackage{amsmath}  % extended mathematics
\usepackage{booktabs} % book-quality tables
\usepackage{units}    % non-stacked fractions and better unit spacing
\usepackage{multicol} % multiple column layout facilities
\usepackage{lipsum}   % filler text
\usepackage{fancyvrb} % extended verbatim environments
  \fvset{fontsize=\normalsize}% default font size for fancy-verbatim environments

\usepackage{color,soul} % Highlights for text


\include{mathcommands.extratex}
\setlength{\extrarowheight}{12pt}

\begin{document}

\definecolor{darkgreen}{rgb}{0.0627, 0.4588, 0.1451}

\maketitle% this prints the handout title, author, and date

\begin{abstract}
\noindent

Denna fil ger lösningar på uppgifterna i modelltentan för kursen.
\end{abstract}

\section{Fråga 1a}

Vi tolkar höger led som att vi har en grupp av $m + w$ personer, och vi skall välja en samling av $k$ av dessa.

Vänster led tolkar vi som att $m$ av dessa personer är män och $w$ av dessa personer är kvinnor. För att välja totalt $k$ personer kan vi börja med att välja hur många skall vara män, och kalla detta antal för $j$, vilket kan vara mellan $0$ och $k$. Sedan skall vi alltså välja $j$ män, av totalt $m$, och $k - j$ kvinnor, av totalt $w$, för att bilda vår grupp av $j + k - j = k$ personer.

\section{Fråga 1b}

Vi tolkar vänster led som att vi har en grupp av $z$ barn, och $n$ av dessa skall få en glass, och $m$ skall få en godispåse. Det är möjligt att något av barnen får både och -- vi gör två separata val för att välja gruppen som får glass och gruppen som får godis, således multiplikationen.

I höger led tänker vi oss att vi väljer antalet barn som skall få både en glass och godis, och kallar detta antal för $k$. $k$ kan variera fritt mellan $k = 0$, alltså inget barn får både och, till $k = n$, alltså att varje barn som får glass också får godis.\sidenote[][]{Notera här att vi lika gärna hade kunnat välja $m$ som vårt övre summeringsindex, eller $\min n, m$.}

Har vi väl valt hur många som skall få bägge, vet vi att det måste vara totalt $n + m - k$ barn som får någonting, och av dessa måste $n - k$ barn få enbart glass och $m - k$ barn enbart godis. Så vi kan börja med att välja vilka barn som får någonting, vilket kan göras på $\binom{z}{m + n - k}$ sätt, och sedan fördela ut de tre rollerna ``enbart glass'', ``enbart godis'', och ``både glass och godis'' mellan dessa barn, och antalet sätt att göra detta på räknas av multinomialkoefficienten $\binom{n + m - k}{k, n-k, m-k}$.

%\bibliography{references}
%\bibliographystyle{plainnat}

\end{document}
