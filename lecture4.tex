\documentclass[nobib]{tufte-handout}

\title{Föreläsning 4: Fortsättning på inklusion-exklusion $\cdot$ 1MA020}

\author[Vilhelm Agdur]{Vilhelm Agdur\thanks{\href{mailto:vilhelm.agdur@math.uu.se}{\nolinkurl{vilhelm.agdur@math.uu.se}}}}

%\date{15 januari 2023}


%\geometry{showframe} % display margins for debugging page layout

\usepackage{graphicx} % allow embedded images
  \setkeys{Gin}{width=\linewidth,totalheight=\textheight,keepaspectratio}
  \graphicspath{{graphics/}} % set of paths to search for images
\usepackage{amsmath}  % extended mathematics
\usepackage{booktabs} % book-quality tables
\usepackage{units}    % non-stacked fractions and better unit spacing
\usepackage{multicol} % multiple column layout facilities
\usepackage{lipsum}   % filler text
\usepackage{fancyvrb} % extended verbatim environments
  \fvset{fontsize=\normalsize}% default font size for fancy-verbatim environments

\usepackage{color,soul} % Highlights for text

% Standardize command font styles and environments
\newcommand{\doccmd}[1]{\texttt{\textbackslash#1}}% command name -- adds backslash automatically
\newcommand{\docopt}[1]{\ensuremath{\langle}\textrm{\textit{#1}}\ensuremath{\rangle}}% optional command argument
\newcommand{\docarg}[1]{\textrm{\textit{#1}}}% (required) command argument
\newcommand{\docenv}[1]{\textsf{#1}}% environment name
\newcommand{\docpkg}[1]{\texttt{#1}}% package name
\newcommand{\doccls}[1]{\texttt{#1}}% document class name
\newcommand{\docclsopt}[1]{\texttt{#1}}% document class option name
\newenvironment{docspec}{\begin{quote}\noindent}{\end{quote}}% command specification environment

\include{mathcommands.extratex}

\begin{document}

\maketitle% this prints the handout title, author, and date

\begin{abstract}
\noindent
Vi fortsätter studera inklusion-exklusion, och ger fler tillämpningar.
\end{abstract}

\section{Surjektioner}

\begin{definition}
  Låt $A$ och $B$ vara två mängder, och $f: A \to B$ en funktion. Vi definierar \emph{bilden} av $A$ som
  $$f(A) = \left\{b \in B \given \exists a\in A: f(a) = b\right\},$$
  det vill säga alla element i $B$ som träffas av något element i $A$ under $f$.

  Funktionen $f$ är en \emph{surjektion} om $f(A) = B$. Om det finns en surjektion från $A$ till $B$ gäller det att $\abs{A} \geq \abs{B}$.\sidenote[][]{Detta är uppenbart för ändliga mängder $A$ och $B$ -- för oändliga mängder är detta definitionen av ordningen mellan kardinaltal.}
\end{definition}

\begin{definition}
  För $n \geq m \geq 1$ ges \emph{Stirlings partitionstal}, också kallat \emph{Stirlingtalet av andra sorten}, av
  $$\stirlingPart{n}{m} = \frac{1}{m!}\sum_{k=0}^{m}(-1)^k\binom{m}{k}(m-k)^n.$$
\end{definition}

\begin{theorem}
  Låt $A$ och $B$ vara ändliga mängder med $\abs{A} = n$, $\abs{B} = m$, och $n \geq m$. Antalet surjektioner från $A$ till $B$ ges av
  $$S(n,m) = m!\stirlingPart{n}{m} = \sum_{k=0}^{m} (-1)^k \binom{m}{k}(m-k)^n.$$

  \begin{proof}
    Låt $X$ vara mängden av alla funktioner från $A$ till $B$, och för varje $b \in B$, låt $X_b$ vara mängden av funktioner från $A$ till $B$ som inte träffar $b$. Vi vill, som vanligt, räkna ut $\abs{X \setminus \bigcup_{b\in B} X_b} = \abs{X} - \abs{\bigcup_{b\in B} X_b}$.

    Multiplikationsprincipen ger oss enkelt att $\abs{X} = m^n$ -- varje element i $A$ har $m$ val för var det skall skickas, och vi har $n$ stycken element att göra det valet för.

    Inklusion-exklusion ger oss att
    $$\abs{\bigcup_{b\in B} X_b} = \sum_{I \subseteq B} (-1)^{\abs{I} + 1}\abs{\bigcap_{b \in I} X_b}$$
    och vad vi behöver räkna är antalet funktioner från $A$ till $B$ som undviker att träffa en viss mängd $I$. Ett specialfall ser vi omedelbart -- om $I = B$ måste snittet vara tomt, eftersom elementen i $A$ måste skickas \emph{någonstans}.

    Att räkna dem är relativt enkelt -- en funktion från $A$ till $B$ som inte träffar en viss mängd $I \subset B$ är ju precis en funktion från $A$ till $B \setminus I$, och vi vet att det finns $\abs{B \setminus I}^{\abs{A}} = (m - \abs{I})^n$ sådana. Så vad vi får är att
    $$\abs{\bigcup_{b\in B} X_b} = \sum_{I \subset B, I \neq B} (-1)^{\abs{I} + 1}(m - \abs{I})^n.$$

    Så om vi grupperar den här summan efter storleken på $I$ vet vi att det finns $\binom{m}{k}$ stycken val av $I$ av storlek $k$, så 
    $$\abs{\bigcup_{b\in B} X_b} = \sum_{k=0}^{m-1} (-1)^{k + 1}\binom{m}{k}(m - k)^n$$
    vilket ger oss resultatet, när vi stoppar in detta i $S(n,m) = \abs{X} - \abs{\bigcup_{b\in B} X_b}$.
  \end{proof}
\end{theorem}

\begin{example}
  Antag att en farmor stickat fem filtar åt sina tre barnbarn. På hur många sätt kan hon fördela filtarna, så att varje barn får åtminstone en filt? Eftersom de är handstickade är så klart filtarna \emph{särskiljbara}, så det här är inte ett exempel på de kompositioner vi såg i föreläsning två, utan ett exempel på surjektioner.

  Vår sats säger oss att svaret är $3!\stirlingPart{5}{3} = 150$.
\end{example}

\section{Övningar}


%\bibliography{references}
%\bibliographystyle{plainnat}

\end{document}
