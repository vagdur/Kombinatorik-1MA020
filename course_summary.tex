\documentclass[nobib]{tufte-handout}

\title{Sammanfattning av hela kursen $\cdot$ 1MA020}

\author[Vilhelm Agdur]{Vilhelm Agdur\thanks{\href{mailto:vilhelm.agdur@math.uu.se}{\nolinkurl{vilhelm.agdur@math.uu.se}}}}

%\date{20 februari 2023}


%\geometry{showframe} % display margins for debugging page layout

\usepackage{graphicx} % allow embedded images
  \setkeys{Gin}{width=\linewidth,totalheight=\textheight,keepaspectratio}
  \graphicspath{{graphics/}} % set of paths to search for images
\usepackage{amsmath}  % extended mathematics
\usepackage{booktabs} % book-quality tables
\usepackage{units}    % non-stacked fractions and better unit spacing
\usepackage{multicol} % multiple column layout facilities
\usepackage{lipsum}   % filler text
\usepackage{fancyvrb} % extended verbatim environments
  \fvset{fontsize=\normalsize}% default font size for fancy-verbatim environments

\usepackage{color,soul} % Highlights for text

% Standardize command font styles and environments
\newcommand{\doccmd}[1]{\texttt{\textbackslash#1}}% command name -- adds backslash automatically
\newcommand{\docopt}[1]{\ensuremath{\langle}\textrm{\textit{#1}}\ensuremath{\rangle}}% optional command argument
\newcommand{\docarg}[1]{\textrm{\textit{#1}}}% (required) command argument
\newcommand{\docenv}[1]{\textsf{#1}}% environment name
\newcommand{\docpkg}[1]{\texttt{#1}}% package name
\newcommand{\doccls}[1]{\texttt{#1}}% document class name
\newcommand{\docclsopt}[1]{\texttt{#1}}% document class option name
\newenvironment{docspec}{\begin{quote}\noindent}{\end{quote}}% command specification environment

\include{mathcommands.extratex}

%\let\emph\relax % there's no \RedeclareTextFontCommand
%\DeclareTextFontCommand{\emph}{\bfseries}

\definecolor{light-gray}{gray}{0.9}
\definecolor{dark-red}{rgb}{0.8, 0, 0}
\definecolor{dark-orange}{rgb}{0.98, 0.69, 0.03}

\usepackage{tikz}
\usetikzlibrary{calc}
\usetikzlibrary{decorations.pathmorphing}

\makeatletter

\newcommand{\defhighlighter}[3][]{%
  \tikzset{every highlighter/.style={color=#2, fill opacity=#3, #1}}%
}

\defhighlighter{yellow}{.5}

\newcommand{\highlight@DoHighlight}{
  \fill [ decoration = {random steps, amplitude=1pt, segment length=15pt}
        , outer sep = -15pt, inner sep = 0pt, decorate
        , every highlighter, this highlighter ]
        ($(begin highlight)+(0,8pt)$) rectangle ($(end highlight)+(0,-3pt)$) ;
}

\newcommand{\highlight@BeginHighlight}{
  \coordinate (begin highlight) at (0,0) ;
}

\newcommand{\highlight@EndHighlight}{
  \coordinate (end highlight) at (0,0) ;
}

\newdimen\highlight@previous
\newdimen\highlight@current

\DeclareRobustCommand*\highlight[1][]{%
  \tikzset{this highlighter/.style={#1}}%
  \SOUL@setup
  %
  \def\SOUL@preamble{%
    \begin{tikzpicture}[overlay, remember picture]
      \highlight@BeginHighlight
      \highlight@EndHighlight
    \end{tikzpicture}%
  }%
  %
  \def\SOUL@postamble{%
    \begin{tikzpicture}[overlay, remember picture]
      \highlight@EndHighlight
      \highlight@DoHighlight
    \end{tikzpicture}%
  }%
  %
  \def\SOUL@everyhyphen{%
    \discretionary{%
      \SOUL@setkern\SOUL@hyphkern
      \SOUL@sethyphenchar
      \tikz[overlay, remember picture] \highlight@EndHighlight ;%
    }{%
    }{%
      \SOUL@setkern\SOUL@charkern
    }%
  }%
  %
  \def\SOUL@everyexhyphen##1{%
    \SOUL@setkern\SOUL@hyphkern
    \hbox{##1}%
    \discretionary{%
      \tikz[overlay, remember picture] \highlight@EndHighlight ;%
    }{%
    }{%
      \SOUL@setkern\SOUL@charkern
    }%
  }%
  %
  \def\SOUL@everysyllable{%
    \begin{tikzpicture}[overlay, remember picture]
      \path let \p0 = (begin highlight), \p1 = (0,0) in \pgfextra
        \global\highlight@previous=\y0
        \global\highlight@current =\y1
      \endpgfextra (0,0) ;
      \ifdim\highlight@current < \highlight@previous
        \highlight@DoHighlight
        \highlight@BeginHighlight
      \fi
    \end{tikzpicture}%
    \the\SOUL@syllable
    \tikz[overlay, remember picture] \highlight@EndHighlight ;%
  }%
  \SOUL@
}
\makeatother

%\renewcommand\emph[1]{{\color{dark-red} \highlight[light-gray]{#1}}}

\renewcommand\emph[1]{\highlight[dark-orange]{#1}}

\begin{document}

\definecolor{darkgreen}{rgb}{0.0627, 0.4588, 0.1451}

\maketitle% this prints the handout title, author, and date

\begin{abstract}
\noindent
Detta dokument ger en sammanfattning av kursens innehåll, med \emph{nyckelord} markerade, och saker vi \emph{räknat} eller \emph{bevisat}.
\end{abstract}

Kursen är uppdelad i tre delar -- vi började med \emph{grundläggande kombinatorik} i de första fyra föreläsningarna, sedan introducerade vi \emph{genererande funktioner} i de kommande tre föreläsningarna. Sedan hade vi ett intermezzo om \emph{grafer} och \emph{träd} i en föreläsning, innan vi fortsatte till vår tredje del om \emph{diskret sannolikhetsteori och den probabilistiska metoden}.

\section{Del ett: Grundläggande kombinatorik}

I den första föreläsningen introducerade vi de allra mest grundläggande koncepten i kombinatoriken:
\begin{enumerate}
    \item \emph{Additionsprincipen} och \emph{multiplikationsprincipen} låter oss räkna olika mängder.
    \item \emph{Ord} bildade ur olika alfabeten är det mest basala av alla kombinatoriska objekt.
    \item Ett viktigt exempel på en slags ord är \emph{permutationer} -- vi definierar och \emph{räknar dessa}.
    \item Om ord är det mest basala exemplet där ordning spelar roll är \emph{kombinationer} det mest grundläggande exemplet på när vi väljer saker utan ordning.
    \item Vi definierar \emph{binomialkoefficienterna} och \emph{visar att} dessa räknar antalet kombinationer av en viss storlek.
\end{enumerate}

Precis i slutet av föreläsning ett börjar vi prata om \emph{kombinatoriska bevis}. I föreläsning två fortsätter vi på detta tema, och ger ett antal olika exempel.

\begin{enumerate}
    \item De flesta av våra \emph{kombinatoriska bevis involverar binomialkoefficienter}, alltså delmängder till en viss mängd i en kombinatorisk tolkning.
    \item Vi \emph{bevisar} specifikt \emph{binomialsatsen} med ett kombinatoriskt bevis.
    \item Sedan definierar vi \emph{omordningar} och använder dessa för att räkna \emph{multi-delmängder}\sidenote[][]{Just termen multi-delmängd introducerar vi tyvärr först i en senare föreläsning -- i efterhand borde termen ha dykt upp redan här. Den refererar till ett sätt att fördela ut $n$ osärskiljbara objekt till $k$ särskiljbara personer, om vi inte kräver att varje person måste få ett objekt.} med ett \emph{pinnar-och-stjärnor-argument}.
    \item Vi ser vårt första exempel av att \emph{räkna lösningar till ekvationer} när vi tolkar en multi-delmängd som en lösning på en ekvation $x_1 + x_2 + \ldots + x_n = k$ -- detta kommer dyka upp igen senare i kursen, med fler begränsningar på vad variablerna kan ta för värden.
    \item Vi definierar \emph{multinomialkoefficienterna}, och ser att dessa ger \emph{antalet omordningar av ett ord}.
\end{enumerate}

I den tredje föreläsningen i denna del av kursen introducerar vi några till enkla verktyg inom kombinatoriken.

\begin{enumerate}
    \item \emph{Lådprincipen}, i dess generaliserade form, låter oss visa en del överraskande resultat. Vi ger ett par enkla exempel, och ett lite mer sofistikerat.
    \item \emph{Inklusion-exklusion} låter oss räkna många saker som annars vore väldigt svåra att räkna. För att kunna bevisa den introducerar vi \emph{indikatorfunktioner} och ger några räkneregler för dessa.
    \item Vi \emph{använder} inklusion-exklusion för att räkna lösningar till ekvationer, nu med övre begränsningar på variablerna.
    \item Vi definierar \emph{derangemang}, och använder inklusion-exklusion för att räkna dessa.
\end{enumerate}

Föreläsning fyra sammanfattar till slut vad vi gjort i denna del av kursen.

\begin{enumerate}
    \item Vi definierar \emph{Stirlings partitionstal}, och \emph{använder} inklusion-exklusion för att visa att antalet \emph{surjektioner} från en mängd till en annan räknas av en formel som involverar dessa.
    \item Vi definierar \emph{mängdpartitioner} och \emph{visar} att dessa räknas av Stirlings partitionstal. Dessa ger oss ett till vanligt exempel på något vi kan ge \emph{kombinatoriska bevis} kring.
    \item Vi skriver upp en stor tre-gånger-fyra tabell över många av de räkneproblem vi sysslat med hittills -- den \emph{tolvfaldiga vägen} -- som sammanfattar och systematiserar det hela i termer av \emph{särskiljbara och osärskiljbara objekt} och funktioner som kan vara \emph{generella, injektiva, eller surjektiva}.
    \item Vi definierar \emph{Stirlings cykeltal}, och därmed också \emph{cykler i permutationer}. Vi \emph{visar} hur man kan \emph{omvandla} mellan en permutation i vanlig form och en i cykelform.
\end{enumerate}

\section{Del två: Genererande funktioner}

\section{Intermezzo: Grafer och träd}

\section{Del tre: Diskret sannolikhetsteori och den probabilistiska metoden}

%\bibliography{references}
%\bibliographystyle{plainnat}

\end{document}
