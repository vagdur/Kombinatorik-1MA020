\documentclass{tufte-handout}

\title{Föreläsning 1: Permutationer och kombinationer $\cdot$ 1MA020}

\author[Vilhelm Agdur]{Vilhelm Agdur\thanks{\href{mailto:vilhelm.agdur@math.uu.se}{\nolinkurl{vilhelm.agdur@math.uu.se}}}}

%\date{15 januari 2023}


%\geometry{showframe} % display margins for debugging page layout

\usepackage{graphicx} % allow embedded images
  \setkeys{Gin}{width=\linewidth,totalheight=\textheight,keepaspectratio}
  \graphicspath{{graphics/}} % set of paths to search for images
\usepackage{amsmath}  % extended mathematics
\usepackage{booktabs} % book-quality tables
\usepackage{units}    % non-stacked fractions and better unit spacing
\usepackage{multicol} % multiple column layout facilities
\usepackage{lipsum}   % filler text
\usepackage{fancyvrb} % extended verbatim environments
  \fvset{fontsize=\normalsize}% default font size for fancy-verbatim environments

\usepackage{color,soul} % Highlights for text

% Standardize command font styles and environments
\newcommand{\doccmd}[1]{\texttt{\textbackslash#1}}% command name -- adds backslash automatically
\newcommand{\docopt}[1]{\ensuremath{\langle}\textrm{\textit{#1}}\ensuremath{\rangle}}% optional command argument
\newcommand{\docarg}[1]{\textrm{\textit{#1}}}% (required) command argument
\newcommand{\docenv}[1]{\textsf{#1}}% environment name
\newcommand{\docpkg}[1]{\texttt{#1}}% package name
\newcommand{\doccls}[1]{\texttt{#1}}% document class name
\newcommand{\docclsopt}[1]{\texttt{#1}}% document class option name
\newenvironment{docspec}{\begin{quote}\noindent}{\end{quote}}% command specification environment

\include{mathcommands.extratex}

\begin{document}

\maketitle% this prints the handout title, author, and date

\begin{abstract}
\noindent
Vi börjar med att fråga oss vad kombinatorik ens är för något. Sedan introducerar vi några väldigt grundläggande begrepp och principer i ämnet, och tillämpar dem på att diskutera permutationer och kombinationer.
\end{abstract}

\section{Vad är kombinatorik?}

Jag hörde en gång, på en fest under min masterutbildning, en utläggning av en doktorand om att all matematik handlar om att reducera sina problem till en enklare form -- och i slutändan var alla matematikproblem antingen linjär algebra, i vilket fall de var lätta, eller så var de kombinatorik, i vilket fall de var svåra. Vi skall alltså studera den svåra delen av matematiken.

En annan överförenklande kategorisering av matematiken ges oss av Randall Munroe.\cite{XKCD_math_classification} Kombinatorik sysslar med den mellersta sortens problem -- där det är lätt att förstå frågan, och inga märkliga kontinuerliga objekt är involverade, men svaret ändå kan vara komplicerat att ta reda på.

\begin{figure}[h]
	\includegraphics{graphics/unsolved_math_problems.png}
	%\caption{Tre typer av olösta matematikproblem}
\end{figure}

En mer ordboksmässig definition av vad kombinatorik är vore att säga att det handlar om att räkna saker, när sakerna är ändligt många och diskreta. Detta är dock heller ingen precis eller uttömmande definition, så det finns saker som är kombinatorik utan att nödvändigtvis handla om att räkna saker, till exempel inom grafteori.

\section{Varför studera kombinatorik?}

Kombinatorik har som redan nämnts tillämpningar i ren matematik -- många problem inom andra grenar av matematiken kan reduceras till problem i kombinatorik. Det har också otaliga tillämpningar utanför den rena matematiken:
\begin{enumerate}
	\item Nätverk och grafer
	\item Analys av algoritmer
	\item Design av kretskort
	\item Design av experiment
\end{enumerate}
Merparten av alla pussel-spel av typen sudoku, eller ``flytta bilarna för att få ut en specifik bil'', etc., kan ses som rena kombinatorikproblem.

\section{Additions- och multiplikations-reglerna}

\begin{definition}[Additions-regeln]
	Om $A$ är en mängd av $n$ objekt och $B$ är en mängd av $m$ objekt så finns det $n+m$ sätt att välja ett objekt från $A$ \emph{eller} ett objekt från $B$. Eller formulerat i symboler, om $\abs{A} = n$ och $\abs{B} = m$ så är $\abs{A \coprod B} = n + m$.\sidenote{Symbolen $\coprod$ betyder \emph{disjunkt union} -- vi tar unionen av de två mängderna, men vi tvingar mängderna att vara disjunkta genom att komma ihåg vilken mängd varje element kom från.  Så om $A$ och $B$ inte har några gemensamma element är det samma sak som $\cup$, men om de har gemensamma element gäller att t.ex. $\{1,2,3\}\cup \{3,4\} = \{1,2,3,4\}$, emedan $\{1,2,3\} \coprod \{3,4\} = \{(1,A), (2,A), (3,A), (3,B), (4, B)\}$, så de har alltså olika antal medlemmar.

För det allra mesta behöver man inte vara så här rigorös, men det kan vara bra att ha i bakhuvudet att summa-regeln inte räknar antalet element i $A\cup B$ om $A$ och $B$ kan tänkas ha gemensamma element. Vad vi gör i det fallet kommer vi återkomma till senare, när vi diskuterar inklusion-exklusion.}
\end{definition}

\begin{example}\label{example_addition_rule}
En restaurang har en meny med fyra drinkar, fem förrätter, tio huvudrätter, och tre desserter. Hur många saker har de på menyn?

Additions-regeln säger oss att svaret är $4+5+10+3 = 22$.
\end{example}

\begin{definition}[Multiplikations-regeln]
	Om $A$ är en mängd av $n$ objekt och $B$ är en mängd av $m$ objekt så finns det $nm$ sätt att välja ett objekt från $A$ \emph{och} ett objekt från $B$. Eller ekvivalent, det finns $nm$ sätt att välja ett par av ett objekt ur $A$ och ett objekt ur $B$. Eller uttryckt i symboler
$$\abs{A\times B} = \abs{\{(a,b) \given a \in A, b \in B\}} = nm.$$
\end{definition}

\begin{example}
	Om du besöker restaurangen i Exempel \ref{example_addition_rule}, hur många olika sätt finns det att beställa en trerätters middag med en drink till?
\sidenote{En fullständigt teoretisk fråga, eftersom ingen faktiskt har råd med det i dagens ekonomi.} Multiplikations-regeln säger oss att svaret är $4\times 5\times 10\times 3 = 600$.
\end{example}

\section{Strängar}

\begin{definition}
En \emph{sträng} $s$ (eller ett \emph{ord}) av längd $n$ på en mängd $X$ (kallad \emph{alfabetet} för strängen) är en funktion
$$s: \{1, 2, \ldots, n\} = [n] \to X$$
där $s_i$ är den $i$te bokstaven i ordet.\sidenote{Från och med nu kommer vi konsekvent använda notationen $[n]$ för mängden av tal mellan $1$ och $n$.}
Vi skriver detta oftast som $s = x_1x_2\ldots x_n$, där $x_i = s(i)$.
\end{definition}

\begin{example}[Binära strängar]
Låt $X = \{0,1\}$. Strängar $s: [n] \to X$ kallas för \emph{binära strängar}. Det finns $2^n$ strängar av längd $n$.\sidenote{Så det finns till exempel åtta binära strängar av längd tre, nämligen
$$000, 001, 010, 011, 100, 101, 110, 111.$$}

Hur vet vi detta? Det finns två val för varje bokstav, så multiplikationsregeln säger oss att det måste finnas $2\times 2\times\ldots\times 2 = 2^n$ att göra ett val av vad varje bokstav skall vara.
\end{example}

\begin{example}[$m$-ära strängar]
Låt $X = \{0,1,\ldots,m-1\}$. Strängar med detta alfabetet kallas för $m$-ära strängar. Om $m = 2$ är de binära, om $m = 3$ är de ternära. 
\end{example}

För generella $X$ kallar vi en sträng $s: [n] \to X$ för en $X$-sträng.

\section{Permutationer}

\section{Kombinationer}

\bibliography{references}
\bibliographystyle{plainnat}



\end{document}
