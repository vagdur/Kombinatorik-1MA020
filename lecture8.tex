\documentclass[nobib]{tufte-handout}

\title{Föreläsning 8: Träd och heltalspartitioner $\cdot$ 1MA020}

\author[Vilhelm Agdur]{Vilhelm Agdur\thanks{\href{mailto:vilhelm.agdur@math.uu.se}{\nolinkurl{vilhelm.agdur@math.uu.se}}}}

%\date{7 februari 2023}


%\geometry{showframe} % display margins for debugging page layout

\usepackage{graphicx} % allow embedded images
  \setkeys{Gin}{width=\linewidth,totalheight=\textheight,keepaspectratio}
  \graphicspath{{graphics/}} % set of paths to search for images
\usepackage{amsmath}  % extended mathematics
\usepackage{booktabs} % book-quality tables
\usepackage{units}    % non-stacked fractions and better unit spacing
\usepackage{multicol} % multiple column layout facilities
\usepackage{lipsum}   % filler text
\usepackage{fancyvrb} % extended verbatim environments
  \fvset{fontsize=\normalsize}% default font size for fancy-verbatim environments

\usepackage{color,soul} % Highlights for text

% Standardize command font styles and environments
\newcommand{\doccmd}[1]{\texttt{\textbackslash#1}}% command name -- adds backslash automatically
\newcommand{\docopt}[1]{\ensuremath{\langle}\textrm{\textit{#1}}\ensuremath{\rangle}}% optional command argument
\newcommand{\docarg}[1]{\textrm{\textit{#1}}}% (required) command argument
\newcommand{\docenv}[1]{\textsf{#1}}% environment name
\newcommand{\docpkg}[1]{\texttt{#1}}% package name
\newcommand{\doccls}[1]{\texttt{#1}}% document class name
\newcommand{\docclsopt}[1]{\texttt{#1}}% document class option name
\newenvironment{docspec}{\begin{quote}\noindent}{\end{quote}}% command specification environment

\include{mathcommands.extratex}

\begin{document}

\definecolor{darkgreen}{rgb}{0.0627, 0.4588, 0.1451}

\maketitle% this prints the handout title, author, and date

\begin{abstract}
\noindent
Vi introducerar grafer och träd, och bevisar att antalet rotade ordnade binära oetiketterade träd också räknas av Catalantalen.

Sedan introducerar vi heltalspartitioner, och härleder en genererande funktion för dessa, som vi använder för några exempel.
\end{abstract}

\section{Grafer och träd}

Vårt första ämne i denna föreläsning är grafer och träd, som kommer dyka upp igen och igen också i senare föreläsningar -- det är ju till och med den preliminära titeln på vår sista föreläsning. Vi börjar med att ge en samling definitioner av vad vi menar med dessa ord, och sedan börjar vi räkna hur många av olika typer av graf det finns i olika klasser.

\begin{definition}
    En \emph{graf} består av en mängd $V$ av \emph{noder} och en mängd $E \subseteq \binom{V}{2}$ av kanter.\sidenote[][]{Med notationen $\binom{A}{k}$ där $A$ är en mängd och $k$ ett heltal menar vi \emph{mängden} av delmängder av storlek $k$ till $n$. Alltså har vi att
    $$\abs{\binom{[n]}{k}} = \binom{n}{k}.$$} Om det finns en kant $\{u,v\}$ säger vi att $u$ och $v$ är \emph{grannar}. En graf är \emph{etiketterad} om noderna är särskiljbara, annars är den oetiketterad.
    \sidenote[][]{Det här är precis samma koncept som med våra lådor som var särskiljbara eller inte. Antingen har noderna namn, så vi kan prata om nod nummer tre, eller så kan vi bara se vilka andra noder de har kanter till.} 
    Vi säger att en graf är \emph{sammanhängande} om det går att nå varje nod från varje annan nod genom att vandra längs kanterna. Ett sätt att vandra från en nod tillbaka till sig själv kallar vi för en \emph{cykel}.

    \begin{figure}
        \centering
        \includegraphics[width=0.4\textwidth]{graphics/example_graph.png}
        \caption[][2cm]{Ett exempel på en graf. Den är inte sammanhängande, eftersom de övre två noderna inte kan nås från de undre fem. Triangeln utgör en cykel, som är grafens enda.}
    \end{figure}
\end{definition}

\begin{example}
    Det finns $2^{\binom{n}{2}}$ stycken etiketterade grafer på $n$ noder, eftersom det finns $\binom{n}{2}$ möjliga kanter, och vi får en graf per val av vilka kanter som skall vara med.

    Problemet med att räkna antalet oetiketterade grafer på $n$ noder är betydligt mer komplicerat. Den första idén man hade haft är kanske att det borde vara
    $$\frac{2^{\binom{n}{2}}}{n!}$$
    eftersom det borde finnas $n!$ olika sätt att sätta dit etiketterna. Problemet är att vissa grafer har symmetrier som gör att till synes olika sätt att skriva dit etiketter i själva verket ger samma etiketterade graf.

    \begin{figure}
        \centering
        \includegraphics[width=0.5\textwidth]{graphics/counting_labelled_graphs.png}
        \caption{Två till synes olika etiketteringar av samma graf, som i själva verket är samma etikettering på grund av grafens rotationssymmetri.}
    \end{figure}

    Som tur är visar det sig att nästan alla grafer inte har någon symmetri alls, så svaret är \emph{nästan} $\frac{2^{\binom{n}{2}}}{n!}$.\sidenote[][-1.5cm]{Det här påståendet låter kanske löst i kanten, men det är faktiskt helt rigoröst. I alla fall om man ersätter ``nästan'' med att skriva att antalet är
    $$(1 + o(1))\frac{2^{\binom{n}{2}}}{n!}.$$}
\end{example}

\begin{definition}
    \begin{marginfigure}
        \centering
        \includegraphics[width=0.6\textwidth]{graphics/example_tree.png}
        \caption{Ett träd med sju noder och sex kanter.}
    \end{marginfigure}

    Ett \emph{träd} är en sammanhängande graf utan cykler. Ett \emph{rotat} träd är ett träd med en specifik nod utpekad som dess rot.\sidenote[][]{Så om trädet är oetiketterat kan vi alltså se vilken nod som är roten, men resten av noderna kan vi inte se skillnad på, bara vilka som hänger ihop med vilka med kanter.} I ett rotat träd har varje nod utom roten själv en granne som är närmre roten än sig\sidenote[][]{Eller är roten.}, vilken vi kallar dess \emph{förälder}. Alla dess andra grannar kallar vi dess \emph{barn}. En nod utan barn kallar vi för ett \emph{löv}, och en nod som inte är ett löv kallar vi för \emph{intern}.

    Ifall det spelar roll i vilken ordning vi ritat noderna kallar vi trädet \emph{ordnat}, se figur \ref{fig:distinct_only_as_ordered}.

    \begin{marginfigure}
        \centering
        \includegraphics[width=0.8\textwidth]{graphics/ordered_versus_unordered_trees.png}
        \caption{Två träd som är olika varandra som ordnade träd, men samma träd som oordnade träd.}
        \label{fig:distinct_only_as_ordered}
    \end{marginfigure}
\end{definition}

\begin{definition}
    Ett träd i vilket alla noder antingen har två eller noll barn kallas för ett \emph{binärt} träd.
\end{definition}

Låt oss nu återse en gammal vän, Catalantalen.

\begin{proposition}
    Antalet rotade ordnade binära oetiketterade träd med $n$ stycken interna noder ges av Catalantalen.

    \begin{proof}
        Vi kan dela upp ett sådant träd i två mindre träd genom att helt enkelt ta bort roten, och låta dess två barn vara rötter i två mindre träd.

        \begin{figure}
            \centering
            \includegraphics[width=0.5\textwidth]{graphics/RUBOTree_division.png}
            \caption[][1.5cm]{Ett rotat ordnat binärt oetiketterat träd, med uppdelningen av det i två mindre träd av samma typ, ett rött och ett blått.}
        \end{figure}

        Alltså gäller det, om $t_n$ betecknar antalet sådana träd, att
        $$t_{n+1} = \sum_{k=0}^{n} t_k t_{n-k},$$
        eftersom vi kan skapa oss ett sådant träd med $n+1$ noder genom att först rita roten, och sedan fästa ett träd med $k$ interna noder till vänster och ett med $n-k$ interna noder till höger. Eftersom roten själv är intern har vi då $k + n - k + 1 = n + 1$ interna noder.
    \end{proof}
\end{proposition}

\section{Cayleys formel}

Vi har alltså lyckats räkna en väldigt specifik sorts träd. Kan vi räkna träd mer generellt?

\begin{theorem}
    Det finns
    $$n^{n-2}$$
    stycken etiketterade träd med $n$ noder.
\end{theorem}

För att förstå detta resultat, låt oss börja med att räkna de första små fallen. Vi kollar på \emph{oetiketterade} träd, och räknar hur många sätt vi kan sätta etiketter på dem.

\begin{figure}
    \centering
    \includegraphics[width = \textwidth]{graphics/counting_tree_labellings.png}
    \caption{Oetiketterade grafer med $n$ noder, för $n = 1,\ldots,5$, med antalet sätt att sätta etiketter på varje i blått.}
\end{figure}

Sätten vi får dessa antal är, per värde på $n$:
\begin{enumerate}
    \item Att det bara finns ett sätt att skriva en etta på den enda noden är uppenbart.
    \item Vi kan välja vilken permutation som helst av $[2]$ att skriva på noderna, men grafen har en speglingssymmetri, så att skriva etiketterna i motsatt ordning ger samma träd. Alltså $\frac{2!}{2}$.
    \item Vi kan se vilken nod som är den mellersta, men vi kan inte se skillnad på de två yttre. Alltså är det enda val vi kan göra det av vilket tal vi skriver på den mellersta, vilket vi kan välja på tre sätt.
    \item För den första av våra två oetiketterade grafer kan vi skriva vilken permutation av $[4]$ vi vill, men återigen har vi en speglingssymmetri, så att skriva den baklänges ger oss samma etikettering. Alltså $\frac{4!}{2}$.
    
    För den andra är vi i samma situation som vi var i för $n=3$ -- vi kan se vilken nod det är som har mer än en granne, men vi kan inte se skillnad på de andra. Alltså är det enda valet vi har vilken etikett just den särskilda noden får, vilket vi kan göra på $4$ sätt.
    \item Vi får $\frac{5!}{2}$ av samma speglingssymmetri-skäl som innan, och för den tredje av våra grafer får vi $5$ eftersom vi åter har en särskild nod och resten kan vi inte se skillnad på.
    
    För den mellersta av våra tre oetiketterade träd kan vi se skillnad på de tre noderna i svansen till vänster, men de två som sticker ut åt höger kan vi inte se skillnad på. Så för att etikettera denna väljer vi två etiketter för de talen, vilket vi kan göra på $\binom{5}{2}$ sätt, och sedan är varje permutation av de återstående tre etiketterna faktiskt en distinkt etikettering, så vi kan välja $3!$ sätt att fullfölja vår etikettering. Så vi har totalt $\binom{5}{2}3!$ sätt att göra detta på.
\end{enumerate}

\section{Övningar}

\begin{xca}
    Bevisa att ett träd alltid har $\abs{E} = \abs{V} - 1$.
\end{xca}

%\bibliography{references}
%\bibliographystyle{plainnat}

\end{document}
