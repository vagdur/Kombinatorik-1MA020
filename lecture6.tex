\documentclass[nobib]{tufte-handout}

\title{Föreläsning 6: Fortsättning på genererande funktioner $\cdot$ 1MA020}

\author[Vilhelm Agdur]{Vilhelm Agdur\thanks{\href{mailto:vilhelm.agdur@math.uu.se}{\nolinkurl{vilhelm.agdur@math.uu.se}}}}

%\date{15 januari 2023}


%\geometry{showframe} % display margins for debugging page layout

\usepackage{graphicx} % allow embedded images
  \setkeys{Gin}{width=\linewidth,totalheight=\textheight,keepaspectratio}
  \graphicspath{{graphics/}} % set of paths to search for images
\usepackage{amsmath}  % extended mathematics
\usepackage{booktabs} % book-quality tables
\usepackage{units}    % non-stacked fractions and better unit spacing
\usepackage{multicol} % multiple column layout facilities
\usepackage{lipsum}   % filler text
\usepackage{fancyvrb} % extended verbatim environments
  \fvset{fontsize=\normalsize}% default font size for fancy-verbatim environments

\usepackage{color,soul} % Highlights for text

% Standardize command font styles and environments
\newcommand{\doccmd}[1]{\texttt{\textbackslash#1}}% command name -- adds backslash automatically
\newcommand{\docopt}[1]{\ensuremath{\langle}\textrm{\textit{#1}}\ensuremath{\rangle}}% optional command argument
\newcommand{\docarg}[1]{\textrm{\textit{#1}}}% (required) command argument
\newcommand{\docenv}[1]{\textsf{#1}}% environment name
\newcommand{\docpkg}[1]{\texttt{#1}}% package name
\newcommand{\doccls}[1]{\texttt{#1}}% document class name
\newcommand{\docclsopt}[1]{\texttt{#1}}% document class option name
\newenvironment{docspec}{\begin{quote}\noindent}{\end{quote}}% command specification environment

\include{mathcommands.extratex}

\begin{document}

\definecolor{darkgreen}{rgb}{0.0627, 0.4588, 0.1451}

\maketitle% this prints the handout title, author, and date

\begin{abstract}
\noindent
Vi fortsätter förra föreläsningens diskussion om genererande funktioner, och ger fler exempel och sätt att använda sådana för att lösa kombinatoriska problem.
\end{abstract}

\section{Antal lösningar till en ekvation, med begränsningar}

I slutet på förra föreläsningen studerade vi antalet lösningar till ekvationen
$$x_1 + x_2 + x_3 + x_4 + x_5 = k$$
om vi kräver att alla $x_i$ är ickenegativa heltal. Det var ett första exempel på en mer generell kategori av problem med att räkna lösningar på ekvationer. Låt oss börja med ett lite mer invecklat problem:

\begin{example}
    Hur många lösningar finns det till
    $$x_1 + x_2 + x_3 + x_4 = k$$
    om vi kräver att alla $x_i$ är ickenegativa heltal, men också kräver att $x_2$ är jämnt, att $x_3 \leq 10$, och $x_4$ är udda?

    Låt, för varje $k$, $a_k$ vara antalet sådana lösningar.
    Låt sedan $a_k^1$ vara antalet lösningar till $x_1 = k$ i ickenegativa heltal $x_1$,
    $a_k^2$ vara antalet lösningar till $x_2=k$ i ickenegativa jämna heltal,
    $a_k^3$ vara antalet lösningar till $x_3=k$ i heltal mellan $0$ och $10$,
    och $a_k^4$ vara antalet lösningar till $x_4 = k$ i udda heltal.

    Precis som i förra exemplet studerar vi nu faltningen av dessa fyra följder, och ser att
    $$(a^1 * a^2 * a^3 * a^4)_k = \sum_{\substack{k_1, k_2, k_3, k_4\geq 0\\k_1 + k_2 + k_3 + k_4 = k}} a_{k_1}^1a_{k_2}^2a_{k_3}^3a_{k_4}^4 = a_k$$
    eftersom $a_{k_1}^1a_{k_2}^2a_{k_3}^3a_{k_4}^4$ är en produkt av ettor och nollor -- att $k_1 + k_2 + k_3 + k_4 = k$ garanteras av definitionen av faltning, och sedan är varje term i produkten ett om värdet på $k_i$ är tillåtet av våra begränsningar, och noll annars. Så produkten är ett om summan är korrekt och varje enskild begränsning är uppfylld.

    Så precis som i förra exemplet kan vi få fram genererande funktionen för $a_k$, följden vi faktiskt är intresserade av, genom att plocka fram den genererande funktionen för de enklare följderna.

    Vad genererande funktionen för $a^1$ är vet vi sedan innan -- den är bara en följd av ettor, så dess genererande funktion blir $\frac{1}{1-x}$. Likaledes vet vi sedan innan att följden av $n$ stycken ettor och sedan nollor har genererande funktion $\frac{1 - x^{n+1}}{1-x}$, så genererande funktionen för $a^3$ blir $\frac{1 - x^{11}}{1-x}$.

    Däremot för $a^2$ behöver vi räkna ut något nytt, nämligen den genererande funktionen för följden $1,0,1,0,1,\ldots$, indikatorfunktionen av de jämna talen. Så vi får skriva att
    \begin{align*}
        F_{a^2}(x) &= \sum_{k=0}^{\infty} a^2_k x^k\\
        &= \sum_{\substack{k \geq 0\\k \in 2\Z}} x^k\\
        &= \sum_{i=0}^{\infty} x^{2i}\\
        &= \sum_{i=0}^{\infty} (x^2)^i
    \end{align*}
    och sista raden här kan vi känna igen som genererande funktionen av följden $(1,1,1,1,\ldots)$, \emph{utvärderad i} $x^2$. Så detta är lika med $\frac{1}{1-x^2}$.

    Så vad som återstår är alltså $a^4$, indikatorfunktionen för de udda talen. För att få fram dess genererande funktion kan vi använda vad vi just gjorde för de jämna talen:
    \begin{align*}
        F_{a^4}(x) &= \sum_{k=0}^{\infty} a_k x^k\\
        &= \sum_{\substack{k \geq 1\\k\text{ udda}}} x^k\\
        &= x\sum_{\substack{k \geq 1\\k\text{ udda}}} x^{k-1}\\
        &= x\sum_{\substack{k \geq 0\\k \in 2\Z}} x^k\\
        &= \frac{x}{1 - x^2}.
    \end{align*}

    Så, om vi använder att genererande produkten av en faltning är produkten av de genererande funktionerna, ser vi att
    \begin{align*}
        F_a(x) &= \left(\frac{1}{1-x}\right)\left(\frac{1-x^{11}}{1-x}\right)\left(\frac{1}{1-x^2}\right)\left(\frac{x}{1-x^2}\right)\\
        &= \frac{x(1 - x^{11})}{(1-x)^2(1-x^2)^2}
    \end{align*}
    och ber vi vårt favorit-CAS\sidenote[][]{\emph{Computer Algebra System}, alltså till exempel \emph{WolframAlpha} eller något av dess öppna alternativ, såsom \emph{Sage}.} att Taylorutvidga detta uttryck så får vi att
    $$F_a(x) = x + 2x^2 + 5x^3 + 8x^4 + 14x^5 + 20x^6 + 30x^7 + 40x^8 + \ldots$$
    så att följden av antalet lösningar är
    $$0,1,2,5,8,14,20,30,40,55,70,91,111,138,163,\ldots.$$
\end{example}

\begin{example}
  Vi vill räkna antalet lösningar $a_k$ till ekvationen
  $$2x_1 + x_2 + x_3 = k$$
  där alla $x_i$ är heltal, $x_2$ är en multipel av $6$, och talet $x_3$ kan vara antingen rött eller blått.\sidenote[][]{Vi ser alltså, för $k = 6$, alla dessa som godtagbara distinkta lösningar:
  $$x_1 = 1, x_2 = 0, x_3 = \textcolor{blue}{4}, \qquad x_1 = 1, x_2 = 0, x_3 = \textcolor{red}{4},$$
  $$x_1 = 2, x_2 = 0, x_3 = \textcolor{blue}{2}, \qquad x_1 = 0, x_2 = 6, x_3 = \textcolor{red}{0}.$$}

  Vi börjar med att göra variabelbytet $y_1 = 2x_1$, och vill alltså nu ha lösningar till $y_1 + x_2 + x_3 = k$, med begränsningen att $y_1$ är jämnt. Det här förändrar så klart inte antalet lösningar, bara gör det lättare för oss att tillämpa vår metod.

  Vi tillämpar samma metod som i förra exemplet, och låter $a^1_k$ vara antalet lösningar till $y_1 = k$ med $y_1$ jämnt, $a^2_k$ vara antalet lösningar till $x_2 = k$ med $x_2$ delbart med $6$, och $a^3_k$ vara antalet lösningar till $x_3 = k$ med $x_3$ färgat antingen rött eller blått. Faltningen blir då 
  $$(a^1 * a^2 * a^3)_k = \sum_{\substack{k_1, k_2, k_3 \geq 0\\k_1+k_2+k_3 = k}} a^1_{k_1}a^2_{k_2}a^3_{k_3} = a_k.$$

  Vi fortsätter precis som innan med att räkna ut den genererande funktionen för varje av våra följder. För $a^1$ vet vi redan vad genererande funktionen för indikatorfunktionen av de jämna talen är, nämligen $\frac{1}{1-x^2}$. 
  
  För $a^2$ kan vi använda samma metod som vi använde för de jämna talen för att se att
  \begin{align*}
    F_{a^2}(x) &= \sum_{k=0}^{\infty} a^2_k x^k = \sum_{\substack{k \geq 0\\k \in 6\Z}} x^k\\
    &= \sum_{i=0}^{\infty} x^{6i} = \sum_{i=0}^{\infty} (x^6)^i
  \end{align*}
  så att $F_{a^2}(x) = \frac{1}{1-x^6}$.

  För $a^3$ så blir denna helt enkelt en följd av bara tvåor, eftersom vi har två val för färg för varje tal, och kan välja vilket tal som helst. Så vi ser att
  $$F_{a^3}(x) = \sum_{k=0}^{\infty} 2x^k = 2\frac{1}{1-x}.$$

  Sammantaget har vi alltså att
  $$F_a(x) = \left(\frac{1}{1-x^2}\right)\left(\frac{1}{1-x^6}\right)\left(\frac{2}{1-x}\right) = \frac{2}{(1-x)(1-x^2)(1-x^6)}$$
  vilket vi kan Taylorutvidga i vårt favoritprogram och få att\sidenote[][]{Vi ser ju ett tydligt mönster här av att $a_{2k} = a_{2k+1}$. Kan du förklara varför detta måste vara fallet, baserat på våra begränsningar av variablerna?}
  \begin{align*}
    F_a(x) &= 2 + 2 x + 4 x^2 + 4 x^3 + 6 x^4 + 6 x^5 + 10 x^6 + 10 x^7\\
    & + 14 x^8 + 14 x^9 + 18 x^{10} + 18 x^{11} + 24 x^{12} + 24 x^{13}\\
    & + 30 x^{14} + 30 x^{15} + 36 x^{16} + 36 x^{17} + 44 x^{18} + \ldots.
  \end{align*}
\end{example}

\begin{example}
  Antag att vi har ett schackbräde med $2\times n$ rutor, och vi vill täcka det med brickor av formen $2\times 1$ eller $1\times 2$. Hur många sätt kan vi göra detta på?

  \begin{marginfigure}
    \includegraphics{graphics/checkerboard_tiling.png}
    \caption{Ett sätt att göra detta då $n=8$. Figur tagen ur förra årets anteckningar.}
  \end{marginfigure}

  Låt $t_n$ vara antalet sätt vi kan täcka vårt schackbräde. Vi vill hitta en rekursion för detta antal. Vi ser enkelt att det finns ett enda sätt att göra det för $n=1$ -- bara en bricka får plats -- så $t_1 = 1$. För $n=2$ finns det två sätt, antingen lägger vi dem horisontellt eller vertikalt, så $t_2 = 2$.

  \begin{marginfigure}
    \includegraphics{graphics/checkerboard_tiling_n_2.png}
    \caption{De två sätten att göra det på då $n=2$. Figur från förra årets föreläsningsanteckningar.}
  \end{marginfigure}

  Om vi vill skapa oss en täckning av en $2\times n$-bräda, för något $n > 2$, kan vi göra på två sätt:
  \begin{itemize}
    \item Vi börjar med en täckning av en $2\times(n-1)$-bräda, och lägger till en till bricka vertikalt.
    \item Vi börjar med en täckning av en $2\times(n-2)$-bräda, och lägger till två till brickor horisontellt.
  \end{itemize}

  Att varje täckning av en $2\times n$-bräda kan skapas på detta vis är enkelt att se -- antingen är den sista kolumnen täckt av en vertikal bricka, i vilket fall vi skapade täckningen på första viset, eller så är den täckt av två horisontella brickor, i vilket fall vi skapade den på det andra sättet.

  Alltså har vi funnit följande rekursion för antalet täckningar
  $$t_0 = 0, t_1 = 1, t_2 = 2, \quad t_n = t_{n-1} + t_{n-2} \, \forall n > 2$$
  som ju är extremt lik den vi har för Fibonaccitalen, så vi kan finna en genererande funktion och sluten form på precis samma vis som i det fallet.
\end{example}

\section{Exponentiella genererande funktioner}

Ibland får vi problem med att hitta enkla uttryck för våra genererande funktioner, eftersom vår följd växer för snabbt. Hittills har vi bara studerat följder som växer långsamt nog, men om vi till exempel hade velat studera följden $0!, 1!, 2!,\ldots$ hade dess vanliga genererande funktion varit 
$$\sum_{k=0}^{\infty} k!x^k$$
för vilken det inte finns något enkelt uttryck.\sidenote[][]{Om vi sätter på oss våra analytiker-glasögon kan vi dessutom se att detta uttryck inte konvergerar för något $x > 0$, så vad hade vi ens kunna ha för uttryck för en funktion som är oändlig överallt?}

Ett annat exempel på detta är om vi räknar permutationer -- dessa kommer vara ungefär $k!$ stycken, i de flesta av våra exempel. Så vi hade haft samma problem. Det finns en anledning att vi hittills bara studerat binomialkoefficienter, som ju är betydligt mindre.

Så, låt oss definiera en variant på genererande funktioner som kan hantera dessa snabbväxande följder.

\begin{definition}
  Om $\{a_k\}_{k=0}^\infty$ är någon följd ges dess \emph{exponentiella genererande funktion} av
  $$EG_a(x) = \sum_{k=0}^{\infty} a_k \frac{x^k}{k!}.$$
\end{definition}

Så allt vi faktiskt har gjort är att skala ner vår följd så att den inte växer så kraftigt. Som tur är fungerar detta enkla trick väldigt väl -- låt oss räkna några exempel för att se hur.

\begin{example}
  Den exponentiella genererande funktionen för följden $(1,1,1,1,\ldots)$ ges av
  $$\sum_{k=0}^{\infty} \frac{x^k}{k!} = e^x.$$
\end{example}

\begin{example}
  Den exponentiella genererande funktionen för följden $(0!, 1!, 2!, 3!, \ldots)$ ges av
  $$\sum_{k=0}^{\infty} k!\frac{x^k}{k!} = \sum_{k=0}^{\infty} x^k = \frac{1}{1-x}.$$
\end{example}

\begin{example}
  Fixera något heltal $n$, och låt $a_k$ vara antalet permutationer av längd $k$ ur ett alfabete med $n$ bokstäver, så att $a_k = \frac{n!}{(n-k)!}$. Då ges den exponentiella genererande funktionen för $a$ av
  \begin{align*}
    EG_a(x) &= \sum_{k=0}^{\infty} \frac{n!}{(n-k)!}\frac{x^k}{k!}\\
    &= \sum_{k=0}^{\infty} \frac{n!}{k!(n-k)!}x^k\\
    &= \sum_{k=0}^{\infty} \binom{n}{k} x^k = (1 + x)^n
  \end{align*}
  där vi i sista ledet kände igen en \emph{ordinär} genererande funktion för binomialkoefficienterna.
\end{example}

\section{Övningar}

\begin{xca}
  Hur många heltalslösningar till ekvationen
  $$x_1 + x_2 + x_3 = 578$$
  finns det,\sidenote[][-2cm]{Om ni väl har hittat genererande funktionen för antalet lösningar när vi ersatt $578$ med $k$, så kan ni enkelt få fram svaret med följande kod till \emph{WolframAlpha}:
  $$\mathtt{SeriesCoefficient}[f, \{x,0,578\}]$$
  där $f$ då är genererande funktionen ni funnit. Att ange den genererande funktionen och säga att ni sedan plockade fram rätt koefficient ur den med hjälp av ett CAS är den förväntade metoden här.} om vi kräver att
  \begin{itemize}
    \item $x_1 \geq -7$,\sidenote[][]{Ledtråd: Gör ett variabelbyte till $y_1$ för att få den vanliga begränsningen att $y_1 \geq 0$.}
    \item $x_2 \geq 0$ är ett jämnt tal,
    \item och $x_3 \geq 0$ kan vara vilket tal som helst, men om det är jämnt kan det vara rött eller blått, och om det är udda kan det vara gult, grönt, eller lila. 
  \end{itemize}
\end{xca}

%\bibliography{references}
%\bibliographystyle{plainnat}

\end{document}
