\documentclass[nobib]{tufte-handout}

\title{Övningstillfälle 2: Teoretiska övningar $\cdot$ 1MA020}

\author[Vilhelm Agdur]{Vilhelm Agdur\thanks{\href{mailto:vilhelm.agdur@math.uu.se}{\nolinkurl{vilhelm.agdur@math.uu.se}}}}

\date{10 februari 2023}


%\geometry{showframe} % display margins for debugging page layout

\usepackage{graphicx} % allow embedded images
  \setkeys{Gin}{width=\linewidth,totalheight=\textheight,keepaspectratio}
  \graphicspath{{graphics/}} % set of paths to search for images
\usepackage{amsmath}  % extended mathematics
\usepackage{booktabs} % book-quality tables
\usepackage{units}    % non-stacked fractions and better unit spacing
\usepackage{multicol} % multiple column layout facilities
\usepackage{lipsum}   % filler text
\usepackage{fancyvrb} % extended verbatim environments
  \fvset{fontsize=\normalsize}% default font size for fancy-verbatim environments

\usepackage{color,soul} % Highlights for text


\include{mathcommands.extratex}

\begin{document}

\maketitle% this prints the handout title, author, and date

\begin{abstract}
\noindent
Detta dokument innehåller en samling övningar i kombinatorik som inte kräver någon programmering, utan är avsedda att lösas med papper och penna. Merparten av dessa problem är svårare än potentiella tentaproblem, eftersom de är avsedda att lösas i grupp över en längre tid, inte snabbt och individuellt i en tentasal.
\end{abstract}


%\bibliography{references}
%\bibliographystyle{plainnat}

\end{document}
